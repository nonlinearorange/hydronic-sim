\subsection{Radiacón Solar}

\subsubsection{El Sol}

\subsubsection{La Constante Solar}

\subsubsection{Distribución Espectral de Radiación Extraterrestre}

\subsubsection{Variación de Radiación Extraterrestre}

\subsubsection{Dirección del Haz de Radiación}

Las relaciones geométricas de cualquier plano con respecto a cualquier orientación relativa a la tierra, en cualquier momento y un haz entrante de radiación solar, pueden ser descritas por varios ángulos, estos por convención son nombrados y se hace referencia hacia ellos como se muestra a continuación,

\begin{itemize}
	\item $\phi$ \textbf{Latitud}, es la ubicación angular al norte o sur del ecuador, siendo la dirección del norte positiva; $-90^\circ\leq\phi\leq90^\circ$.
	\item $\delta$ \textbf{Declinación}, es la posición angular del sol al medio día solar con respecto al plano del ecuador, por ejemplo, cuando este se encuentra en la meridiana local. Esta es positiva al norte; $-23.45^\circ\leq\delta\leq23.45^\circ$.
	\item $\beta$ \textbf{Pendiente}, es el ángulo entre el plano de la superficie en cuestión y la superficie; $0^\circ\leq\beta\leq180^\circ$.
	\item $\gamma$ \textbf{Ángulo de Acimut de Superficie}, es la desviación de la proyección en un plano horizontal a la superficie desde la meridiana local. $-180.0^\circ\leq\gamma\leq180.0^\circ$.
	\item $\omega$ \textbf{Ángulo de Hora}, el desplazamiento angular del sol del este o el oeste con respecto a la meridiana local debido a la rotación de la tierra cuyo valor es de $15^\circ$ por hora; En la mañana es negativo y en el atardecer positivo.
	\item $\theta$ \textbf{Ángulo de Incidencia}, es el ángulo entre el haz de radiación en una superficie y la normal de esa superficie.
	\item $\theta_z$ \textbf{Ángulo Cenital}, es el ángulo entre la vertical y la línea hacia el sol, en otras palabras, es el ángulo de incidencia del haz de radiación en una superficie horizontal.
	\item $\alpha_s$ \textbf{Ángulo de Altitud Solar}, es el ángulo entre la horizontal y la línea hacia el sol, es el complemento del ángulo cenital.
	\item $\gamma_s$ \textbf{Ángulo de Acimut Solar}, es el desplazamiento angular del sur de la proyección del haz de radiación en un plano horizontal. Los desplazamientos del este al sur son negativos y del oeste al sur son positivos.	
\end{itemize}

\paragraph{La Declinación $\delta$}
Este ángulo puede ser encontrado a través de una aproximación como la que se muestra a continuación,
\begin{equation} \label{eq:declinacion_ingenieria}
\delta = 23.45 \sin\left( \frac{360}{365}\left( 284 + n\right) \right) 
\end{equation}
Para fines de ingeniería, la aproximación anterior resulta suficiente, sin embargo disponemos de otro modelo cuya precisión es mayor ya que el error es menor al $0.035^\circ$.

\begin{align} \label{eq:declinacion_precisa}
	\begin{aligned}
		\delta & = 0.006918 - 0.399912\cos\left(\Gamma \right) + 0.070257\sin(\Gamma) \\
		& -0.006758\cos\left(2\Gamma \right)  + 0.000907\sin\left(2\Gamma \right)  \\
		& -0.002697\cos\left(3\Gamma \right) + 0.00148\sin\left(3\Gamma \right)
	\end{aligned}
\end{align}

en dónde,
\begin{equation}
\Gamma = \frac{2\pi\left(n-1 \right) }{365}
\end{equation}

En ambas expresiones la variable $n$ es el número del día del año que se estudia, esta satisface $1\leq n \leq 365$.